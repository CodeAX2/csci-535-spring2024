\documentclass{article}

\usepackage[margin=1in]{geometry}
\usepackage{amsmath}

\usepackage{algorithm,algorithmic}
\renewcommand{\algorithmicrequire}{\textbf{Input:}}
\renewcommand{\algorithmicensure}{\textbf{Output:}}
\renewcommand{\algorithmiccomment}[2]{\hspace{#1}$\triangleright$ {#2} \hfill }

\usepackage{amsfonts}
\def\R{{\mathbb R}}
\def\N{{\mathbb N}}

\title{Topological Sort: Decrementing Functions and Loop Invariants}
\author{CSCI 432}

\begin{document}
\maketitle

\noindent
Name:\\
Who did you work with today?

\section*{Topological Sort: The Algorithm}

Topological sort takes a poset $(P,\leq)$ and returns a total order of all
objects that is \emph{compatible} with the partial order of the poset.  By
compatible, we mean that for all $a \leq b \in P$, we have $a$ appears before
$b$ in the total order.  For example, consider $P$ to be a set of points in
$\R^2$ and let the partial order be the product order; that is, $(a_x,a_y) \leq
(b_x,b_y)$ if and only if $a_x \leq b_x$ and $a_y \leq b_y$.  Given a set:
$$ P = \{ (0,0), (\pi, \pi), (2,4), (1,2) \}$$
we can return the following total order:
$$[ (0,0), (1,2), (2,4), (\pi, \pi)].$$
Note that this is not the only compatible total order.  What other total order
could have been returned?

\textbf{Answer}
\vspace{1in}

The following algorithm takes as input a poset, represented as a directed
acyclic graph, and returns a compatible total order.  If the poset orders $a
\leq b$, then the DAG has an edge from vertex $a$ to vertex $b$ (so, pointing
from `smaller' to `larger' here).  For this
algorithm, vertices have at least one attribute, a variable named outgoing, which stores a list
of vertices that define the outgoing edges. We present the algorithm on the next
page.

\begin{algorithm}[h]
    \caption{Topological Sort}
    %
    \begin{algorithmic}[1]
        \REQUIRE $G=(V,E)$, a DAG

        \ENSURE $L$, a list of vertices in G in a total order compatible with
        the partial order of the DAG.

        \FOR {$v \in V$}
            \STATE Add an attribute `$v$.count' and initialize it to the indegree of $v$.
        \ENDFOR

        \STATE $readyV \gets$ the set of vertices with `count'=0.\label{ln:initQ}

        \STATE $L \gets$ array of length $|V|$
        \STATE $i \gets 1$

        \WHILE {$readyV$ is not empty}
            \STATE $v \gets readyV.\textsc{POP}$
            \STATE $L[i] \gets v$
            \STATE $i++$
            \FOR {$u \in v$.outgoing}
                \STATE $u$.count$--$
                \IF {$u$.count $=0$}
                    \STATE Add $u$ to $readyV$
                \ENDIF
            \ENDFOR
        \ENDWHILE

        \RETURN $L$
    \end{algorithmic}
\end{algorithm}

\pagebreak
In Line~\ref{ln:initQ}, why is $readyV$ not empty? (Hint: you need to prove
this!)

\textbf{Answer}
\vspace{1in}

Explain how topological sort is helpful for dynamic programming.

\textbf{Answer}
\vspace{1in}

Walk through a small example. (This should always be one of the first things you
do when you encounter a new algorithm).

\textbf{Answer}

\pagebreak
\section*{Topological Sort: The Decrementing Function}

We can prove that an algorithm terminates in two ways: (1) prove it directly by
giving an (asymptotic) runtime; (2) find a decrementing function.  A
decrementing function is a function from the state space to a well-ordered set
(usually the natural numbers, $\N$).  There is one rule though: the function
MUST decrement between iterations of a loop or when making a recursive call. If
such a function exists, then we know that the algorithm terminates!

What is the decrementing function for the for loop?

\textbf{Answer}
\vspace{1in}

What is the decrementing function for the while loop?

\textbf{Answer}
\vspace{1in}

\section*{Topological Sort: The Loop Invariant}

Now that we know that the algorithm terminates, let's prove that it is correct.
Here, we investigate partial correctness of the while loop. (It's called partial
correctness because we are proving "If the algorithm terminates, then it is
correct").

First, we need to identify a few statements.

For this loop, what
are the following statements (start with your best guess, then come back and
revise if needed):

First, the loop guard is the statement that keeps us in the loop.
What is the loop guard $G$?
\vspace{0.5in}

The pre-conditions are everything that we know to be true upon entering the loop
for the first time.
What is the pre-condition $P$?
\vspace{0.5in}

The post-condition is what we want to be true at the end of the loop.  (i.e.,
what is the loop supposed to do?)
What is the post-condition $Q$?
\vspace{0.5in}

This one is tricky and might take some iterations.
The loop invariant is the statement that we need in order to prove
Initialization, Maintenance, and End below. Sometimes it might be helpful to
include $i$, the number of times that we've iterated through the loop, in the
notation for the loop invariant.  We call it an invariant because it's always
true.
What is the loop invariant $L=L_i$?
\vspace{0.5in}


There are three things that we need to prove.

INITIALIZATION: The loop invariant must be true the first time we attempt
entering the loop.  We abbreviate this as: $P \implies L$.

\textbf{Proof}
\vspace{2in}

MAINTENANCE: The loop invariant must be true each time we attempt to enter the
loop.  So, if we enter the loop and the loop invariant is true, then the loop
invariant must be true the next time we attempt to enter the loop.  We
abbreviate this as $L_i \wedge G \implies L_{i+1}$.

\textbf{Proof}
\vspace{3in}

END: OK, we've entered the loop and the loop invariant is true. Then, we
re-enter with the loop invariant true. Over and over and over again.  Finally,
we exit the loop.  We now need to prove that if we exit the loop (and the loop
invariant is true), then we've done what we wanted to do! In short, $L \wedge
\neg G \implies Q$.

\textbf{Proof}
\vspace{2in}

\end{document}
