\documentclass[14pt,leqno]{article}
\usepackage{amsmath,amssymb,amsthm,amsfonts}
\usepackage{tikz}
\usepackage{multicol}
\pagestyle{plain}
\renewcommand{\baselinestretch}{1.0}

\setlength{\textwidth}{6.5in}
\setlength{\oddsidemargin}{0.249in}
\setlength{\evensidemargin}{0.25in}
\setlength{\topmargin}{-0.5in}
\setlength{\textheight}{9.51in}



\begin{document}




\begin{enumerate}

\item $\textbf{\v{C}ech vs Rips complex}$

\begin{enumerate}
\item Show that $\textbf{\v{C}ech} (F,r)  \subset \textbf{Rips} (F,2.r)$, where $r \ge 0$

\item  Show that $ \textbf{Rips} (F,r) \subset \textbf{\v{C}ech} (F,r)$, where $r \ge 0$

\item  Show that $ \textbf{Rips} (F, \sqrt{2} \cdot r ) \subset \textbf{\v{C}ech} (F,r)$ , where $r \ge 0$. (This can be done by considering the geometry of the standard simplex. For a short cut, you can look  up Jung's theorem in Euclidean spaces)

\end{enumerate}


\item  $\textbf{A convexity-like condition in the nerve theorem is necessary}$

[\textit{In the proof of the nerve theorem, we use the fact that the covering sets are closed and convex (as discussed, contractible - or even less - is enough). This exercise shows that such assumptions are needed.}]

Sketch a collection of  five subsets of $\mathbb{R}^2$ whose nerve is not homotopy equivalent to their union.

\item  $\textbf{alpha  complex}$

[\textit{Given a finite number of points $x_1, x_2 \dots x_n$ on a subset $\Omega$ of the plane, we define $V_{i} = \{ x \in \Omega : \|x - x_i\| \leqslant \|x - x_j\| \text{ for all } j \neq i\}$. A Voronoi diagram with seeds $x_1, x_2 \dots x_n$ is the mosaic of $\Omega$ by the regions $V_1, V_2 \dots V_n $. As discussed in class, the following exercise - along with the nerve theorem - shows that $\textbf{\v{C}ech} (F,r)$ and $\textbf{alpha} (F,r)$ have the same homotopy type.}] \\

Show that, for $r \ge 0$, $\displaystyle{\bigcup_{i=1}^n \{B(x_i,r) \cap V_i \} = \bigcup_{i=1}^n B(x_i,r) }  $




\end{enumerate}








\end{document}
